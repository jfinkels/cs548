\documentclass{article}
\usepackage{fullpage}
% new commands
\newcommand{\collaborators}[1]{\emph{Collaborators: #1}}

% author, date, and title
\author{Jef{}frey~Finkelstein}
\date{26 January 2012}
\title{CS548 problem set 1}

\begin{document}
\maketitle
\collaborators{Davide, Dimitris, Xianrui}
\begin{enumerate}
\item
  The interactive proof for graph isomorphism is a proof of knowledge.
  Construct a knowledge extractor $E$ as follows on input $(G_0, G_1)$, proceeding in steps from stage $i=1$ up to $n$.
  \begin{enumerate}
  \item The prover sends the intermediate graph $H$ (a graph which is isomorphic to $G_0$ under a random isomorphism, $\sigma$) to the extractor.
  \item The extractor lets $b=0$ and sends it to the prover, which sends back an isomorphism $\psi_0$ (which, if honest, should be $\sigma$).
  \item The extractor lets $b=1$ and sends it to the prover, which sends back an isomorphism $\psi_1$ (which, if honest, should be $\pi\circ\sigma$).
  \item For both values of $b\in\{0,1\}$, check if $\psi_b$ is a correct isomorphism from $G_b$ to $H$ (checking whether a given permutation is a correct isomorphism can be done in polynomial time).
  \item
    There are now three cases.
    \begin{enumerate}
    \item If both are correct, then $\psi_0\circ\psi_1^{-1}=\sigma\circ(\pi\circ\sigma)^{-1}=\sigma\circ\sigma^{-1}\circ\pi=\pi$, in which case accept and output $\pi$.
    \item If neither are correct, then reject and output nothing.
    \item
      If exactly one is correct, increment the stage number to $i+1$ and repeat.
      If $i$ is greater than $n$, then reject and output nothing.
    \end{enumerate}
  \end{enumerate}
  This algorithm runs in polynomial time, since each stage is a polynomial time bounded interactive protocol, and it runs for $n$ rounds, where $n$ is the security parameter.
  A polynomial times $n$ is still a polynomial in $n$.

  If this extractor rejects with no output at any stage before stage $n$, then the original verifier would have rejected (since it would have received an incorrect isomorphism from the prover).
  If this extractor rejects with no output at stage $n$, then the original verifier would have accepted with probability at most $\frac{1}{2^n}$ (since there were two cases at each of the $n$ stages in which exactly one of the two isomorphisms was correct).
  If this extractor accepts and outputs an isomorphism, then the probability that the original verifier would have accepted is greater than $\frac{1}{2^n}$ (since it will accept and output on any case other than the previous ones).

\item
  An interactive proof $(P, V)$ is WI if and only if it is ZK with a computationally unbounded simulator.

  ($\leftarrow$) Suppose $(P, V)$ is a zero knowledge proof with respect to a computationally unbounded simulator.
  So for all non-uniform probabilistic polynomial time verifiers $V^*$ there exists a simulator $S$ such that for all $x$ and $w$ with $(x,w)\in R$, and for all $z$, $[P, V^*]\left((x, w), (x, z)\right)\simeq S\left((x, z)\right)$.
  Then for all witnesses $w_1$ and $w_2$ with $(x,w_1)\in R$ and $(x,w_2)\in R$, we have that
  \begin{displaymath}
    [P, V*]\left((x, w_1), (x, z)\right)\simeq S\left((x, z)\right)
  \end{displaymath}
  and
  \begin{displaymath}
    S\left((x, z)\right)\simeq[P, V^*]\left((x, w_2), (x, z)\right).
  \end{displaymath}
  Since the $\simeq$ relation is transitive, it follows that $[P, V^*]\left((x, w_1), (x, z)\right)_2\simeq[P, V^*]\left((x, w_2), (x, z)\right)_2$.
  Therefore the $(P, V)$ interactive proof is witness indistinguishable.

  ($\rightarrow$) I am assuming here that the relation $R$ which the $(P, V)$ interactive protocol decides is an $\mathsf{NP}$ relation.
  Hence the length of a witness of a string $x$ is bounded above by some polynomial $p$ in the length of the string $x$.

 Suppose $(P, V)$ is witness indistinguishable, so for all $x$, $w_1$, and $w_2$ with $(x, w_1)\in R$ and $(x, w_2)\in R$, for all non-uniform probabilistic polynomial time verifiers $V^*$, and for all $z$, $[P, V^*]\left((x, w_1), (x, z)\right)_2\simeq [P, V^*]\left((x, w_2), (x, z)\right)_2$.
  Construct a new simulator $S$ as follows on input $(x, z)$.
  \begin{enumerate}
  \item
    Perform an exhaustive search on strings up to length $p(|x|)$, where $p$ is the polynomial which bounds the lengths of the witnesses of $x$ with respect to the relation $R$, in order to find a witness of $x$ (for example, the lexicographically first witness).
    If no witness is found after searching all strings up to length $p(|x|)$, reject and output nothing.
    Otherwise, call this witness $w_0$.
  \item Output the result of $[P, V^*]\left((x, w_0), (x, z)\right)$.
  \end{enumerate}

  If no witness exists, then $x$ is not in the $\mathsf{NP}$ language induced by $R$, and this simulator always rejects.
  To show correctness, suppose $(x, w)\in R$.
  Then $S\left((x, w)\right)=[P, V^*]\left((x, w_0), (x, z)\right)\simeq[P, V^*]\left((x, w), (x, z)\right)$.
  Therefore, the $(P, V)$ interactive proof is zero knowledge.

\item
  Proof by contradiction using a hybrid argument.
  Assume $(P^{\|k}, V^{\|k})$ is not a witness indistinguishable interactive protocol, so there exists a parallel verifier $V^*$ that distinguishes the witness vector $(w^1_1, w^1_2, \ldots, w^1_k)$ from witness vector $(w^2_1, w^2_2, \ldots, w^2_k)$.
  By a hybrid argument, $V^*$ also distinguishes hybrid witness vector $(w^1_1, w^1_2, \ldots, w^1_{i-1}, w^2_i, w^2_{i+1}, \ldots, w^2_k)$ from $(w^1_1, w^1_2, \ldots, w^1_{i-1}, w^1_i, w^2_{i+1}, \ldots, w^2_k)$ for some $i\in\{1, 2, \ldots, k\}$.
  Construct a new verifier $\hat{V}$ which will distinguish witnesses when interacting with the (non-parallel) prover $P$ as follows.
  Suppose $x$ is the common input to the verifier (and the prover $P$).
  The auxiliary non-uniform input to $\hat{V}$ is the vector of common inputs $(x_1, x_2, \ldots, x_k)$, the vector $(w^1_1, w^1_2, \ldots, w^1_{i-1}, w^2_{i+1}, \ldots, w^2_k)$, and the corresponding index $i$.

  For all $j\neq i$, the verifier $\hat{V}$ will simulate the interaction of the parallel verifier $V^*$ with the parallel prover on common input $x_j$ and witness $w_j$.
  For the $i$th interaction, however, $\hat{V}$ will act as an intermediary between the $i$th instance of the parallel verifier $V^*$ and the external, non-parallel prover $P$.
  The verifier $\hat{V}$ will simulate this external interaction in parallel with the rest of the internal interactions between the parallel prover and the parallel verifier (for example, when the parallel verifier instances send a message to the parallel prover instances, the verifier $\hat{V}$ sends that message out to the external non-parallel prover $P$; messages from the internal parallel and external non-parallel provers are synchronized and sent to the internal parallel verifier simultaneously).

  Now $\hat{V}$ can distinguish whether $P$ uses $w^1_i$ or $w^2_i$.
  Finally, $\hat{V}$ runs in polynomial time because the parallel prover and parllel verifier run in polynomial time (with respect to the security parameter $n$), and the overall interaction with the non-parallel prover just involves a simulation among polynomial time protocols.
  Therefore $(P, \hat{V})$ is not a witness indistinguishable interactive protocol.
\end{enumerate}
\end{document}
