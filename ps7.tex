\documentclass[draft]{article}
\usepackage{fullpage}
\usepackage{amsmath}
\usepackage{amssymb}

% new commands
\newcommand{\collaborators}[1]{\emph{Collaborators: #1}}
%\newcommand{\cid}{\overset{c}{\simeq}}
%\newcommand{\ncid}{\overset{c}{\not\simeq}}
\newcommand{\class}[1]{{\ensuremath{\mathsf{#1}}}}
\newcommand{\getr}{\overset{R}{\gets}}
%\newcommand{\getrn}{\getr\{0, 1\}^n}
\newcommand{\getrsingle}{\getr\{0, 1\}}
%% \newcommand{\getrell}{\getr\{0, 1\}^{\ell(n)}}
\newcommand{\getrpoly}{\getrsingle^{poly(n)}}
\newcommand{\NP}{\class{NP}}
%\newcommand{\mim}{\mathsf{mim}}
%\newcommand{\msim}{\mathsf{sim}}

% author, date, and title
\author{Jef{}frey~Finkelstein}
\date{1 May 2012}
\title{CS548 problem set 7}

\begin{document}
\maketitle
\collaborators{Ben, Davide, Dimitris}
\begin{enumerate}
\item
  Proof by contrapositive.
  Assume $\pi$ does not securely emulate $\phi$, so there exists an adversary $A'$ such that for all simulators $S'$ there exists an environment $E'$ and some auxiliary information $z'\in\{0, 1\}^n$ such that the two probability ensembles are computationally distinguishable, for some $n\in\mathbb{N}$.
  Our goal is to show that there exists an adversary $A$ and an environment $E$ such that for all simulators $S$ there exists an auxiliary input $z$ such that the two probability ensembles are computationally distinguishable, for some $n'\in\mathbb{N}$.
  Choose $A=A'$.
  Construct environment $E$ which, on auxiliary input $z=(E', z')$, does exactly what $E'(z')$ does.
  The $E'$ and $z'$ included in the auxiliary input $z$ are the environment and auxiliary input which correspond to the external simulator $S$ with which $E$ will be interacting (in the ideal interaction).
  We know such an $E'$ and $z'$ exist by our initial assumption (that is, for all $S'$ there exist an $E'$ and $z'$, etc.; in this case, $S$ is covered by the universally quantified $S'$).

  Since we chose $A=A'$ and $E'$ and $z'$ to depend on the external simulator $S$, the output of $E(z)$ will be distributed identically to the output of $E'(z')$ in both the real and ideal interactions.
  Since the $(E', \pi, A')$ and $(E', \phi, S)$ probability ensembles are computationally distinguishable with auxiliary input $z'$, so will be the $(E, \pi, A)$ and $(E, \phi, S)$ with auxiliary input $z=(E', z')$.
\item
  \begin{enumerate}
  \item $\mathcal{F}_{OT}$ is defined as follows:
    \begin{enumerate}
    \item Receive $x_0$ and $x_1$ from sender.
    \item Receive $b$ from receiver.
    \item Send $x_b$ to receiver.
    \end{enumerate}
  \item
    Let $(S, R)$ be the oblivious transfer protocol described in class, and let $(G, D, F, B)$ be the enhanced trapdoor permutation.
    To show that $(S, R)$ securely realizes the ideal oblivious transfer functionality described by $\mathcal{F}_{OT}$, we need to show that both the sender and the receiver are simulatable.
    In some places, we will use the notation in \cite{goldreich2}.
    \begin{description}
    \item[Simulator for the sender] \hfill \\
      Let $A_S$ be an adversarial sender.
      The simulator $S_S$ proceeds as follows on input $(x_0, x_1)$:
      \begin{enumerate}
      \item Run $A_S$ on inputs $(x_0, x_1)$ to yield $\alpha$.
      \item $y_0, y_1 \getr D(\alpha)$ (uniformly and independently).
      \item Send $(y_0, y_1)$ to $A_S$, which yields $(c_0, c_1)$.
      \item Send $(x_0, x_1)$ to $\mathcal{F}_{OT}$.
      \end{enumerate}
      The transcript produced by this simulator is distibuted identically to the view of the sender in the real execution:
      \begin{itemize}
      \item the inputs $(x_0, x_1)$ are the same
      \item since $f_\alpha$ is a trapdoor \emph{permutation}, the images in the real execution of the two uniformly randomly distributed elements of the domain $D_\alpha$ are also distributed in a uniformly random way.
      \end{itemize}
      No environment exists which can distinguish the transcript of the ideal execution from the view of the real execution, otherwise it could distinguish between two identical distributions, which is impossible.
      Therefore the transcript and the view are indistinguishable in both the real and the ideal executions.
    \item[Simulator for the receiver] \hfill \\
      Let $A_R$ be an adversarial receiver.
      The simulator $S_R$ proceeds as follows on input $b$:
      \begin{enumerate}
      \item Send $b$ to $\mathcal{F}_{OT}$.
      \item Receive $x_b$ from $\mathcal{F}_{OT}$.
      \item $r\getrpoly$.
      \item $(\alpha, t) \gets G(1^n, r)$.
      \item Send $\alpha$ to $A_R$, which yields $(y_0, y_1)$.
      \item $c_b \gets x_b \oplus HC(f^{-1}_\alpha(y_b))$.
      \item $c_{\bar{b}} \getrsingle$.
      \item Send $(c_0, c_1)$ to $A_R$, which yields a bit $d$.
      \end{enumerate}
      The simulator outputs the transcript of its execution with $A_R$, along with its randomness and inputs.
      With the exception of $c_{\bar{b}}$, the transcript of the simulator is distributed identically to the view of the real interaction: $r$ is generated uniformly at random, $\alpha$ and $t$ are generated according to $G$ which appears random, the messages $y_0$ and $y_1$ are the same as the real adversary's, and $c_b$ is produced in the same way as in the real execution.
      In the simulation, $c_{\bar{b}}$ is uniform and independently random, whereas in the real execution, $c_{\bar{b}}$ is the hard-core bit for $f_\alpha^{-1}(x_{\bar{b}})$.
      Since the trapdoor permutation is enhanced, no distinguisher, even with $\alpha$ and the randomness used to sample $x_{\bar{b}}$ in the real execution, can distinguish between $c_{\bar{b}}$ and random.
    \end{description}
  \item
    We assume that a party can either send its correct input or $\textsf{abort}$ to the trusted party.
    Further, we assume that if a party does not send anything to the trusted party, this is the same as $\textsf{abort}$.
    We modify the ideal functionality so that it proceeds as follows:
    \begin{enumerate}
    \item
      Receive either $(x_0, x_1)$ or $\textsf{abort}$ from sender.
      If $\textsf{abort}$, send $\bot$ to both sender and receiver.
    \item
      Receive either $b$ or $\textsf{abort}$ from receiver.
      If $\textsf{abort}$, send $\bot$ to both sender and receiver.
    \item Send $x_b$ to receiver.
    \end{enumerate}
    In this way, we compensate for any fail-stop adversaries by halting the entire multi-party computation.
  \end{enumerate}
\item
  \begin{enumerate}
  \item $\mathcal{F}_{com}$ is defined as follows:
    \begin{enumerate}
    \item Receive $m$ from committer.
    \item Send $\textsf{committed}$ to receiver.
    \item Receive $\textsf{open}$ from committer.
    \item Send $m$ to receiver.
    \end{enumerate}
  \item
    Suppose we have the following components:
    \begin{itemize}
    \item $(C, R)$ is a computationally hiding and perfectly binding commitment scheme.
    \item $f$ is a one-way function.
    \item
      $(P_c, V_c)$ is a zero-knowledge proof of knowledge for the \NP-relation defined by
      \begin{displaymath}
        R_c=\{(c, (b, \rho)) \,|\, c=commit_C(b, \rho) \},
      \end{displaymath}
      where $b$ is a bit to which $C$ commits, $\rho$ is the randomness for the commitment, and $c$ is the commitment itself.
    \item $(P_f, V_f)$ is a zero-knowledge proof of knowledge for the \NP-relation defined by
      \begin{displaymath}
        R_f=\{(y, x)\,|\,y = f(x)\}.
      \end{displaymath}
    \item $(P', V')$ is a zero-knowledge proof for the \NP-relation defined by
      \begin{displaymath}
        R_0=\{((c, b, y), (\rho, x)) \,|\, c = commit_C(b, \rho) \lor y = f(x)\}.
      \end{displaymath}
    \end{itemize}
    We now construct commitment scheme $(C', R')$ which securely implements the ideal functionality for $\mathcal{F}_{com}$.
    The scheme is defined by the commit phase and the open phase, on input (to the committer) $b$:
    \begin{description}
    \item[Commit phase] \hfill
      \begin{enumerate}
      \item[($R'1$)]
        Choose $x\getrpoly$.
        Let $y = f(x)$.
        Send $y$ to $C'$.

        At this point, $R'$ simulates $P_f$ while $C'$ simulates $V_f$ in a zero-knowledge proof of knowledge that $(y, x)\in R_f$.
      \item[($C'1$)]
        Choose $\rho\getrpoly$.
        Let $c=commit_C(b, \rho)$.
        Send $c$ to $R'$.

        At this point, $C'$ simulates $P_c$ while $R'$ simulates $V_c$ in a zero-knowledge proof of knowledge that $(c, (b, \rho))\in R_c$.
      \end{enumerate}
    \item[Open phase] \hfill \\
      $C'$ simulates $P'$ while $R'$ simulates $V'$ in a zero-knowledge proof that $((c, b, y), (\rho, x))\in R_0$.
    \end{description}
    We first note that even with the addition of the zero-knowledge proofs (of knowledge), this protocol remains a computationally hiding and perfectly binding commitment scheme (whose hardness is based on that of the underlying commitment scheme $(C, R)$).
    If it is not computationally hiding, there is an adversary which can guess, with high probability, the committed bit $b$ given the commitment $c$.
    This violates the computational hiding of the underlying commitment scheme $(C, R)$ (by constructing a new adversary which uses the first adversary by just simulating the rest of the $(C', R')$ protocol as expected).
    If it is not perfectly binding, there is some non-negligible fraction of random coins $r$ for the receiver $R'$ such that some sequence of messages from the committer $C'$ is an ambiguous commitment (that is, possibly a commitment to 0 and possibly a commitment to 1).
    This implies there is some non-negligible fraction of random coins for the underlying receiver $R$ such that some sequence of messages from the underlying committer $C$ is an ambiguous commitment, as well.
    Therefore, $(C', R')$ remains a computationally hiding and perfectly binding commitment scheme.

    To show that $(C', R')$ securely realizes the ideal commitment functionality described by $\mathcal{F}_{com}$, we need to show that both the committer and the receiver are simulatable.
    \begin{description}
    \item[Simulator for the committer] \hfill \\
      Suppose $E_c$ is the proof of knowledge extractor for $(P_c, V_c)$.
      Let $A_{C'}$ be an adversarial committer, and construct a simulator $S_{C'}$ which proceeds as follows:
      \begin{description}
      \item[Commit phase] \hfill
        \begin{enumerate}
        \item $x\getrpoly$.
        \item $y\gets f(x)$.
        \item Send $y$ to $A_{C'}$.
        \item Simulate the interaction of $P_f$ (having input $y$ and witness $x$) with $A_{C'}$, which will yield a commitment $\tilde{c}$.
        \item Simulate the interaction of $E_c$ with $A_{C'}$, which will yield $(\tilde{b}, \tilde{\rho})$.
        \item Send $\tilde{b}$ to $\mathcal{F}_{com}$.
        \end{enumerate}
      \item[Open phase] \hfill
        \begin{enumerate}
        \item Simulate the interaction of $V'$ (having input $(\tilde{c}, \tilde{b}, y)$) with $A_{C'}$, which will yield $\textsf{open}$.
        \item Send $\textsf{open}$ to $\mathcal{F}_{com}$.
        \end{enumerate}
      \end{description}
      The simulator $S_{C'}$ outputs the transcript of its interaction with $A_{C'}$ (including the transcript output by the extractor $E_c$).
      By choosing a random preimage $x$ and by simulating the actions of $P_f$, $E_c$, and $V'$ in the appropriate places, the simulator provides a transcript which is indistinguishable from the view of the adversary when interacting with the real receiver, $R'$.

      Assume now that there is an environment $\mathcal{E}$ which can distinguish between $\textsc{Exec}^\mathcal{E}_{A_{C'}, R'}(z)$ and $\textsc{Exec}^{\mathcal{E}}_{S_{C'}, \mathcal{F}_{com}}(z)$.
      If $f$ is a one-way function, then $\mathcal{E}$ cannot invert it, and hence must distinguish in the $(P_c, V_c)$ phase, or the $(P', V')$ phase of the protocol.
      If it distinguishes in either of these phases, we could construct an adversary which breaks the hiding of the underlying commitment scheme $(C, R)$.
      If $(C, V)$ is a hiding commitment scheme, then the environment must distinguish in the $(P_f, V_f)$ phase, in which case we could construct an adversary which breaks the one-way-ness of $f$.
      Thus no such environment can exist, so for all environments the real and ideal executions are indistinguishable.
    \item[Simulator for the receiver] \hfill
      Suppose $E_f$ is the proof of knowledge extractor for $(P_f, V_f)$.
      Let $A_{R'}$ be an adversarial receiver, and construct a simulator $S_{R'}$ which proceeds as follows:
      \begin{description}
      \item[Commit phase] \hfill
        \begin{enumerate}
        \item Receive $\textsf{committed}$ from $\mathcal{F}_{com}$.
        \item $\rho\getrpoly$.
        \item $c\gets commit_C(0, \rho)$.
        \item Run $A_{R'}$ to yield $y$.
        \item Simulate the interaction of $E_f$ with $A_{R'}$, which will yield $\tilde{x}$ (the preimage of $y$ under $f$).
        \item Send $c$ to $A_{R'}$.
        \item Simulate the interaction of $P_c$ (having input $c$ and witness $(0, \rho)$) with $A_{R'}$.
        \end{enumerate}
      \item[Open phase] \hfill
        \begin{enumerate}
        \item Receive $\textsf{open}$ from $\mathcal{F}_{com}$.
        \item Simulate the interaction of $P'$ (with input $(c, b, y)$ and witness $(\rho, \tilde{x})$) with $A_{R'}$.
        \end{enumerate}
      \end{description}
      The simulator outputs the transcript of its interaction with $A_{R'}$ (including the transcript output by the extractor $E_f$).
      As above, the transcript is computationally indistinguishable from the view of the adversary when interacting with the real committer, $C'$.
      Note that $c$, the commitment to $0$, is indistinguishable from the real commitment because of the computational hiding of the underlying commitment scheme $(C, R)$.

      Assume now that there is an environment $\mathcal{E}$ that can distinguish between $\textsc{Exec}^\mathcal{E}_{A_{R'}, C'}(z)$ and $\textsc{Exec}^{\mathcal{E}}_{S_{R'}, \mathcal{F}_{com}}(z)$.
      The argument here is similar to the argument above, and so is omitted.
      We conclude that no such environment exists, and hence we have shown that for all environments the real and ideal executions are indistinguishable.
    \end{description}
    Since we have shown that both the committer and the receiver for the $(C', R')$ protocol are simulatable in the ideal world, this completes the proof that $(C', R')$ securely realizes $\mathcal{F}_{com}$.
  \end{enumerate}
\end{enumerate}
\bibliographystyle{plain}
\bibliography{bibliography}
\end{document}
