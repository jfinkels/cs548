\documentclass[draft]{article}
\usepackage{fullpage}
\usepackage{amsmath}
\usepackage{amssymb}

% new commands
\newcommand{\collaborators}[1]{\emph{Collaborators: #1}}
%\newcommand{\cid}{\overset{c}{\simeq}}
%\newcommand{\ncid}{\overset{c}{\not\simeq}}
\newcommand{\class}[1]{{\ensuremath{\mathsf{#1}}}}
\newcommand{\getr}{\overset{R}{\gets}}
%\newcommand{\getrn}{\getr\{0, 1\}^n}
\newcommand{\getrsingle}{\getr\{0, 1\}}
%% \newcommand{\getrell}{\getr\{0, 1\}^{\ell(n)}}
\newcommand{\getrpoly}{\getrsingle^{poly(n)}}
\newcommand{\NP}{\class{NP}}
%\newcommand{\mim}{\mathsf{mim}}
%\newcommand{\msim}{\mathsf{sim}}

% author, date, and title
\author{Jef{}frey~Finkelstein}
\date{1 May 2012}
\title{CS548 problem set 7}

\begin{document}
\maketitle
\collaborators{Ben, Davide, Dimitris}
\begin{enumerate}
\item
  Proof by contrapositive.
  Assume $\pi$ does not securely emulate $\phi$, so there exists an adversary $A'$ such that for all simulators $S'$ there exists an environment $E'$ and some auxiliary information $z'\in\{0, 1\}^n$ such that the two probability ensembles are computationally distinguishable, for some $n\in\mathbb{N}$.
  Our goal is to show that there exists an adversary $A$ and an environment $E$ such that for all simulators $S$ there exists an auxiliary input $z$ such that the two probability ensembles are computationally distinguishable, for some $n'\in\mathbb{N}$.
  Choose $A=A'$.
  Construct environment $E$ which, on auxiliary input $z=(E', z')$, does exactly what $E'(z')$ does.
  The $E'$ and $z'$ included in the auxiliary input $z$ are the environment and auxiliary input which correspond to the external simulator $S$ with which $E$ will be interacting (in the ideal interaction).
  We know such an $E'$ and $z'$ exist by our initial assumption (that is, for all $S'$ there exist an $E'$ and $z'$, etc.; in this case, $S$ is covered by the universally quantified $S'$).

  Since we chose $A=A'$ and $E'$ and $z'$ to depend on the external simulator $S$, the output of $E(z)$ will be distributed identically to the output of $E'(z')$ in both the real and ideal interactions.
  Since the $(E', \pi, A')$ and $(E', \phi, S)$ probability ensembles are computationally distinguishable with auxiliary input $z'$, so will be the $(E, \pi, A)$ and $(E, \phi, S)$ with auxiliary input $z=(E', z')$.
\item
  \begin{enumerate}
  \item $\mathcal{F}_{OT}$ is defined as follows:
    \begin{enumerate}
    \item Receive $x_0$ and $x_1$ from sender.
    \item Receive $b$ from receiver.
    \item Send $x_b$ to receiver.
    \end{enumerate}
  \item I don't know.
  \item I don't know.
  \end{enumerate}
\item
  \begin{enumerate}
  \item $\mathcal{F}_{com}$ is defined as follows:
    \begin{enumerate}
    \item Receive $m$ from committer.
    \item Send $\textsf{committed}$ to receiver.
    \item Receive $\textsf{open}$ from committer.
    \item Send $m$ to receiver.
    \end{enumerate}
  \item
    Suppose we have the following components:
    \begin{itemize}
    \item $(C, R)$ is a commitment scheme.
    \item $f$ is a one-way function.
    \item $(P_c, V_c)$ is a zero-knowledge proof of knowledge for the \NP-relation defined by $R_c=\{(c, (b, \rho)) \,|\, c=commit_C(b, \rho) \}$, where $b$ is a bit to which $C$ commits, $\rho$ is the randomness for the commitment, and $c$ is the commitment itself.
    \item $(P_f, V_f)$ is a zero-knowledge proof of knowledge for the \NP-relation defined by $R_f=\{(y, x)\,|\,y = f(x)\}$, where $f\in\mathcal{F}$.
    \item $(P', V')$ is a zero-knowledge proof for the \NP-relation defined by $R'=\{((c, b), (\rho, x)) \,|\, c = commit_C(b, \rho) \lor y = f(x)\}$.
    \end{itemize}
    We now construct commitment scheme $(C', R')$ which securely implements the ideal functionality for $\mathcal{F}_{com}$.
    The scheme is defined by the commit phase and the open phase, on input (to the committer) $b$:
    \begin{description}
    \item[Commit phase] \hfill
      \begin{enumerate}
      \item[($R'1$)]
        Choose $x\getrpoly$.
        Let $y = f(x)$.
        Send $y$ to $C'$.

        At this point, $R'$ simulates $P_f$ while $C'$ simulates $V_f$ in a zero-knowledge proof of knowledge that $(y, x)\in R_f$.
      \item[($C'1$)]
        Choose $\rho\getrpoly$.
        Let $c=commit_C(b, \rho)$.
        Send $c$ to $R'$.

        At this point, $C'$ simulates $P_c$ while $R'$ simulates $V_c$ in a zero-knowledge proof of knowledge that $(c, (b, \rho))\in R_c$.
      \end{enumerate}
    \item[Open phase] \hfill \\
      $C'$ simulates $P'$ while $R'$ simulates $V'$ in a zero-knowledge proof that $((c, b), (\rho, x))\in R'$.
    \end{description}
    We first note that even with the addition of the zero-knowledge proofs (of knowledge), this protocol remains a computationally hiding and binding commitment scheme (whose hardness is based on that of the underlying commitment scheme $(C, R)$).
    Assume it is not computationally hiding.
    Then we can construct an adversary which can guess the committed bit $b$ when receiving the commitment $c$.

    To show that $(C', R')$ securely realizes the ideal commitment functionality described by $\mathcal{F}_{com}$, we need to show that both the committer and the receiver are simulatable.
    \begin{description}
    \item[Simulator for the committer] \hfill \\
      Suppose $E_c$ is the proof of knowledge extractor for $(P_c, V_c)$.
      Let $A_{C'}$ be an adversarial committer, and construct a simulator $S_{C'}$ which proceeds as follows:
      \begin{description}
      \item[Commit phase] \hfill
        \begin{enumerate}
        \item $x\getrpoly$.
        \item $y\gets f(x)$.
        \item Send $y$ to $A_{C'}$.
        \item Simulate the interaction of $P_f$ (having input $y$ and witness $x$) with $A_{C'}$, which will yield a commitment $\tilde{c}$.
        \item Simulate the interaction of $E_c$ with $A_{C'}$, which will yield $(\tilde{b}, \tilde{\rho})$.
        \item Send $\tilde{b}$ to $\mathcal{F}_{com}$.
        \end{enumerate}
      \item[Open phase] \hfill
        \begin{enumerate}
        \item Simulate the interaction of $V'$ (having input $\tilde{c}$) with $A_{C'}$, which will yield $\textsf{open}$.
        \item Send $\textsf{open}$ to $\mathcal{F}_{com}$.
        \end{enumerate}
      \end{description}
      The simulator $S_{C'}$ outputs the transcript of its interaction with $A_C$ (including the transcript output by the extractor $E_c$).
      We will show that this transcript is indistinguishable from the view of the adversary $A_{C'}$ when interacting with the real (honest) receiver $R'$.

      First, in the commit phase, the honest receiver $R'$ sends an image under $f$ of a random element in the domain of $f$; the simulator does the same thing, these random elements of the domain of $f$ are indistinguishable.
      Next, $R'$ engages in a zero-knowledge proof that $(y, x)\in R_f$, acting as a prover; the simulator does the same thing, so the transcript of this interaction will be indistinguishable from the transcript of the interaction between $A_{C'}$ and $R'$.
      Next, the adversary $A_{C'}$ outputs a commitment $\tilde{c}$; since the previous two parts of the interaction are indistinguishable, so will be this commitment.
      Next, $R'$ engages in a zero-knowledge proof that $(c, (b, \rho))\in R_c$, acting as the verifier; the simulator uses the extractor $E_c$ to produce, along with the extracted witness $(\tilde{b}, \tilde{\rho})$, an indistinguishable transcript which the simulator includes in its output.
      Finally, in the open phase, the simulator does exactly what the honest receiver would have done.

      In this ideal execution, the extracted bit $\tilde{b}$ is provided to the ideal functionality, which eventually sends that bit up to the environment.
      In the real execution, the extracted bit


    \item[Simulator for the receiver] \hfill
      Suppose $E_f$ is the proof of knowledge extractor for $(P_f, V_f)$.
      Let $A_{R'}$ be an adversarial receiver, and construct a simulator $S_{R'}$ which proceeds as follows:
      \begin{description}
      \item[Commit phase] \hfill
        \begin{enumerate}
        \item Receive $\textsf{committed}$ from $\mathcal{F}_{com}$.
        \item $\rho\getrpoly$.
        \item $c\gets commit_C(0, \rho)$.
        \item Run $A_{R'}$ to yield $y$.
        \item Simulate the interaction of $E_f$ with $A_R$, which will yield $\tilde{x}$ (the preimage of $y$ under $f$).
        \item Send $c$ to $A_R$.
        \item Simulate the interaction of $P_c$ (having input $c$ and witness $(0, \rho)$) with $A_R$.
        \end{enumerate}
      \item[Open phase] \hfill
        \begin{enumerate}
        \item Receive $\textsf{open}$ from $\mathcal{F}_{com}$.
        \item Simulate the interaction of $P'$ (with input $(c, b)$ and witness $(\rho, \tilde{x})$) with $A_R$.
        \end{enumerate}
      \end{description}
    \end{description}
  \end{enumerate}
\end{enumerate}
\end{document}
