\documentclass[draft]{article}
%\usepackage{fullpage}
\usepackage{amsmath}
\usepackage{amssymb}

% new commands
\newcommand{\collaborators}[1]{\emph{Collaborators: #1}}
\newcommand{\cid}{\overset{c}{\simeq}}
\newcommand{\ncid}{\overset{c}{\not\simeq}}
%% \newcommand{\class}[1]{{\ensuremath{\mathsf{#1}}}}
\newcommand{\getr}{\overset{R}{\gets}}
\newcommand{\getrn}{\getr\{0, 1\}^n}
\newcommand{\getrsingle}{\getr\{0, 1\}}
%% \newcommand{\getrell}{\getr\{0, 1\}^{\ell(n)}}
\newcommand{\getrpoly}{\getrsingle^{poly(n)}}
%% \newcommand{\NP}{\class{NP}}
\newcommand{\mim}{\mathsf{mim}}
\newcommand{\msim}{\mathsf{sim}}

% author, date, and title
\author{Jef{}frey~Finkelstein}
\date{17 April 2012}
\title{CS548 problem set 6}

\begin{document}
\maketitle
\collaborators{Davide, Dimitris, Xianrui}
\begin{enumerate}
\item
  Let $(C, R)$ be a commitment scheme which is non-malleable with respect to identity.
  Let $\mathcal{S}$ be a signature scheme, defined by $\mathcal{S}=(Gen, Sgn, Ver)$, which has existential unforgeability against chosen message attacks.
  Then the following commitment scheme $(C', R')$ is non-malleable with respect to content.
  Construct committer $C'$ which proceeds as follows on input $m$:
  \begin{enumerate}
  \item $(SK, VK) \gets Gen(1^n)$.
  \item $v\gets C(m, VK)$.
  \item $\sigma\gets Sgn(v, SK)$.
  \item Send $(\sigma, v, VK)$ to $R'$.
  \end{enumerate}
  Construct receiver $R'$ which proceeds as follows on input $(\sigma, v, VK)$:
  \begin{enumerate}
  \item If $Ver(\sigma, v, VK)$ rejects, output $\bot$.
  \item $m\gets R(v, VK)$.
  \end{enumerate}

  To show that $(C', R')$ is non-malleable with respect to content, we present a proof by contrapositive.
  Assume that $(C', R')$ is malleable with respect to content, so there exists an adversary $A'$ such that $\mim^{A'}_{(C', R')}(\sigma, v, VK, z)$ is computationally distinguishable from $\mim^{A'}_{(C', R')}(\tilde{\sigma}, \tilde{v}, \tilde{VK}, z)$ (where $z$ is auxiliary input to $A'$).
  We wish to show that either $\mathcal{S}$ is existentially forgeable with a chosen message attack or $(C, R)$ is malleable.
  Suppose $\mathcal{S}$ is existentially unforgeable against chosen message attacks.
  We will show that $(C, R)$ is malleable by constructing an adversary $A$ which proceeds as follows on input $(v, ID)$ and auxiliary input $z_0=(z, SK)$ where $SK$ is the signing key corresponding to $ID$ (interpreted as a verification key):
  \begin{enumerate}
  \item $(v, \sigma)\gets Sgn(v, SK)$.
  \item $(\tilde{v}, \tilde{\sigma}, \tilde{ID})\gets A'(v, \sigma, ID, z)$.
  \item If $ID=\tilde{ID}$, abort.
  \item Output $(\tilde{v}, \tilde{ID})$.
  \end{enumerate}
  First, note that if $ID=\tilde{ID}$ then $A'$ has produced a $\tilde{v}$ and $\tilde{\sigma}$ which can be verified using $ID$ as the verification key, thus breaking existential unforgeability.
  Since we assumed $\mathcal{S}$ is secure, the probability of this happening is negligible, hence the probability of $A$ aborting is negligible.

  Second, note that the only way to get a single signing key $SK$ into the auxiliary input corresponding to each $ID$ is to assume the signature scheme is a one-time signature scheme.

  Suppose $D'$ is the probabilistic polynomial time machine which distinguishes $\mim^{A'}_{(C', R')}(\sigma, v, VK, z)$ from $\mim^{A'}_{(C', R')}(\tilde{\sigma}, \tilde{v}, \tilde{VK}, z)$.
  Construct a distinguisher $D$ which proceeds as follows on input $(v, ID, z)$: run $D'$ on $(v, R, ID, z)$ where $R\getrpoly$, and output whatever $D'$ outputs.
  $D$ will do exactly what $D'$ does on input $(v, R, ID, z)$, so it will distinguish $(v, ID, z)$ from $(\tilde{v}, \tilde{ID}, z)$.

  Therefore we have shown that if $(C', R')$ is malleable, then either $\mathcal{S}$ is existentially forgeable with a chosen message attack or $(C, R)$ is malleable.
  In other words, if $(C, R)$ and $\mathcal{S}$ are secure, then so is $(C', R')$.

\item
  This solution uses the ideas from \cite[Claim~4]{lpv}.

  Suppose $(C, R)$ is the concurrent non-malleable commitment scheme for identifiers of length $\lg n + 1$.
  We construct a one-one concurrent commitment scheme $(C', R')$ for identifiers of length $n$ as follows.
  The committer $C'$ proceeds as follows on input $m$ and identifier $d=d_1d_2\cdots d_n$, both of length $n$:
  \begin{enumerate}
  \item $r_1,\ldots,r_n\getrn$ such that $m=\bigoplus_{i=1}^n r_i$.
  \item For $i\in\{0, 1\}^{\lg n}$ concurrently, $c_i \gets C(r_i, i\circ d_i)$.
  \item Send $((c_1, 1\circ d_1), (c_2, 2\circ d_2), \ldots, (c_n, n\circ d_n))$.
  \end{enumerate}
  The receiver $R'$ proceeds as follows on input $((c_1, 1\circ d_1), \ldots, (c_n, n\circ d_n))$:
  \begin{enumerate}
  \item For $i\in\{0, 1\}^{\lg n}$ concurrently, $r_i\gets R(c_i, i\circ d_i)$.
  \item $m\gets\bigoplus_{i=1}^n r_i$.
  \end{enumerate}

  We claim that if $(C, R)$ is concurrent non-malleable with identifiers of length $\lg n + 1$ then $(C', R')$ is one-one concurrent non-malleable with identifiers of length $n$.
  The proof is by contrapositive.
  Assume $(C', R')$ is malleable, so there exists an adversary $A'$ which meets the definition of malleable.
  We construct an adversary $A$ which proceeds as follows, receiving inputs $(c_1, 1\circ d_1), \ldots, (c_n, n\circ d_n)$ concurrently on the left:
  \begin{enumerate}
  \item $(\tilde{c}_1, \tilde{\mathsf{id}}_1), \ldots, (\tilde{c}_n, \tilde{\mathsf{id}}_n) \gets A'((c_1, 1\circ d_1), \ldots, (c_n, n\circ d_n))$
  \item Send $(\tilde{c}_i, \tilde{\mathsf{id}}_i)$ to receiver instance $i$ for all $i\in\{1,2,\ldots, n\}$.
  \end{enumerate}
  Now $\mim^{A}_{(C, R)}=\mim^{A'}_{(C', R')}\ncid\mim^{A'}_{(C', R')}=\mim^{A}_{(C, R)}$, which shows that the view of $A$ on the left is computationally distinguishable from the view of $A$ on the right.
  Therefore if $(C, R)$ is concurrent non-malleable then $(C', R')$ is one-one concurrent non-malleable.

  I don't know if it's possible to adapt this proof to construct a one-one concurrent non-malleable commitment scheme that allows for identities of length $p(n)$ for arbitrary polynomials $p$.
  One idea is to repeat the above construction on each of $\frac{p(n)}{n}$ blocks (each of size $n$).
  However, to index the identifier $d=d_1d_2\cdots d_{p(n)}$ without repeats would require $\lg p(n)=k\lg n$ bits for some $k$, instead of the $\lg n$ allowed.
\end{enumerate}
\bibliographystyle{plain}
\bibliography{bibliography}
\end{document}
