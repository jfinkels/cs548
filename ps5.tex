\documentclass[draft]{article}
%\usepackage{amsmath}

% new commands
\newcommand{\collaborators}[1]{\emph{Collaborators: #1}}
%% \newcommand{\cid}{\overset{c}{\simeq}}
%% \newcommand{\class}[1]{{\ensuremath{\mathsf{#1}}}}
%% \newcommand{\getr}{\overset{R}{\gets}}
%% \newcommand{\getrn}{\getr\{0, 1\}^n}
%% \newcommand{\getrsingle}{\getr\{0, 1\}}
%% \newcommand{\getrell}{\getr\{0, 1\}^{\ell(n)}}
%% \newcommand{\getrpoly}{\getrsingle^{poly(n)}}
%% \newcommand{\NP}{\class{NP}}

% author, date, and title
\author{Jef{}frey~Finkelstein}
\date{27 March 2012}
\title{CS548 problem set 5}

\begin{document}
\maketitle
\collaborators{Davide, Dimitris, Xianrui}
\begin{enumerate}
\item I don't know.
\item 
  This protocol is a zero-knowledge protocol because it satisfies completeness, soundness, and zero-knowledge.
  We assume the underlying $(P, V)$ protocol has completeness 1, and that the ``bogus'' witness used is the empty string, $\lambda$.
  We will call the first part of the $(P', V')$ protocol, in which $P'$ simulates $V(x)$ and $V'$ simulates $P(x, \lambda)$, ``phase 1'' and the subsequent simulated $(P, V)$ interaction (if it occurs) ``phase 2''.

  \begin{description}
  \item[Completeness] \hfill \\
    Suppose $(x, w)\in R$ (so $w$ is the unique witness for $x$ in $R$).
    First, consider the case in which $w\neq\lambda$.
    In this case, phase 1 will reject with probability 1.
    Then phase 2 of the protocol is identical to the interaction of $P(x, w)$ with $V(x)$, which will accept with probability 1.
    In the case that $w=\lambda$, $P'$ will send $w$ to $V$ after the phase 1.
    At that point the verifier can directly verify that $(x, w)\in R$ in polynomial time with probability 1.
    Therefore, in any case, $V'$ will accept with probability 1.
  \item[Soundness] \hfill \\
    I don't know.
  \item[Zero knowledge] \hfill \\
    Assume $E$ is the proof of knowledge extractor for $(P, V)$ and $S$ is the zero knowledge simulator for $(P, V)$.
    Let $\hat{V}'$ be a cheating verifier for the $(P', V')$ protocol.
    Construct a simulator $S'$ for $\hat{V}'$ protocol which proceeds as follows on input $x$:
    \begin{enumerate}
    \item
      For phase 1, simulate $E(x)$ interacting with $\hat{V}'$ to produce a transcript $T$ and a witness $w'$.
      Technically, $E(x)$ expects to be interacting with the prover $P$ which would work if $\hat{V}'$ were an honest verifier.
      If $\hat{V}'$ does not act like an honest verifier in this phase, the transcript will reflect this, so this simulator must check this.
      That is, the simulator checks $T$ to see if $\hat{V}'$ aborted early or otherwise sent a message which does not match the $(P, V)$ protocol, in which case this simulator aborts.
    \item If $T$ reflects that the underlying $V$ would have accepted in phase 1, send $w'$ to $\hat{V}'$.
    %% TODO But what if \hat{V}' cheats and does not act like $V$ in this step?
    \item Otherwise, simulate $S(x)$ interacting with $\hat{V}'$ in phase 2 to produce a transcript $T$.
    \end{enumerate}

    The simulation of phase 1 is computationally indistinguishable from the interaction with the real prover because the simulator simply runs $E$, and the $(E, P)$ interaction is computationally indistinguishable from the $(V, P)$ interaction.
    In the case that $\hat{V}'$ does not behave like $P$ in phase 1, the $(P, V)$ would have aborted, so the extractor would have aborted with high probability and hence the simulator would have aborted with high probability.

    If the real prover would have sent the witness $w$ to $\hat{V}'$ after phase 1, so will the simulator, since it gets a witness $w'$ using the extractor $E$.
    Since there is at most one witness, $w$ is computationally indistinguishable from $w'$.
    
    If the real prover would have engaged in phase 2, the simulator $S'$ runs $S$, and by the zero-knowledge property of $(P, V)$, the transcript produced by $S$ when interacting with $V$ (which $\hat{V}'$ simulates in this phase) is computationally indistinguishable from the interaction of the real prover $P$ with $V$.

    Therefore, we have shown that the view of $\hat{V}'$ when interacting with $S'$ is computationally indistinguishable from the view of $\hat{V}'$ when interaction with $P$.
  \end{description}

  However, this protocol is not a concurrent zero-knowledge protocol.
  Consider the following adversary which controls two concurrent instances of the verifier $V'(x)$ interacting with $P'(x, w)$ (call them $V'_1$ and $V'_2$):
  \begin{enumerate}
  \item Run the interaction between $V'_1(x)$ and $P'(x, w)$.
  \item If $P'$ sends $w$, do something.
  \item Otherwise simulate the interaction of $V'_1(x)$ (which in phase 2 is playing the part of $V(x)$) with $V'_1(x)$ (which in phase 1 is playing the part of $P(x, \lambda)$).
  \end{enumerate}
\end{enumerate}
\end{document}
