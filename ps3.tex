\documentclass[draft]{article}
\usepackage{amsmath}

% new commands
\newcommand{\collaborators}[1]{\emph{Collaborators: #1}}
\newcommand{\class}[1]{{\ensuremath{\mathsf{#1}}}}
\newcommand{\lang}[1]{{\ensuremath{\mathsf{#1}}}}
\newcommand{\BPP}{\class{BPP}}
\newcommand{\NP}{\class{NP}}
\newcommand{\cid}{\overset{c}{\simeq}}
\renewcommand{\pmod}[1]{\,(\bmod{\,#1})}

% author, date, and title
\author{Jef{}frey~Finkelstein}
\date{16 February 2012}
\title{CS548 problem set 3}

\begin{document}
\maketitle
\collaborators{}
\begin{enumerate}
\item
  We wish to show that if the basic non-interactive zero knowledge protocol of Feige and Shamir is a two-time non-interactive zero knowledge protocol, then $\lang{HAMCYCLE}\in\BPP$ and hence $\NP\subseteq\BPP$.

  In the basic non-interactive zero knowledge protocol of Feige and Shamir for \lang{HAMCYCLE}, the output of the simulator $S$ includes a pair of injective functions from the vertices of the common input graph $G$ to the rows and columns of the Hamiltonian submatrix of the common reference string.
  We construct a $\BPP$ algorithm which runs the simulator $S$ two times (simulating the prover using the same common reference string twice) to decide whether two different graphs have Hamiltonian cycles.
  We use the two pairs of injective functions which $S$ outputs to construct an isomorphism from a graph with a known Hamiltonian cycle to a subgraph of the original input graph $G$.

  Suppose the following:
  \begin{itemize}
  \item The common input for the protocol is a graph $G$ with vertex set $V$ and edge set $E$.
    Let $C_G$ be the Hamiltonian cycle in $G$ (if it exists).
  \item $C_n$ is the \emph{cycle graph} on $n$ vertices, defined by vertex set $\{1, 2, \ldots, n\}$ and edge set $\left\{(i, i \pmod{n} + 1)\,\middle|\, i\in \{1, 2, \ldots, n\}\right\}$.
  \item The prover uses the same common reference string $M$ over two instances of the protocol (with two common input graphs).
  \item $H$ is the Hamiltonian submatrix of the common reference string $M$ and $C_H$ is the corresponding Hamiltonian cycle (if $H$ exists in $M$).
  \item The prover provides a certificate given by a pair of injective functions $(\pi_1, \pi_2)$, where $(u, v)\in C_G$ implies $(\pi_1(u), \pi_2(v))\in C_H$.
  \end{itemize}
  We will construct two algorithms, a helper function $A'$ and a main algorithm $A$ which uses $A'$ as a subroutine, in order to simplify our analysis later.
  Construct the function $A'$ as follows on input $G$:
  \begin{enumerate}
  \item Run $S$ on $G$ to get $T_G$, the transcript which results from the simulation, including $(\phi_1, \phi_2)$, a pair of injective functions describing a mapping from $C_G$ to $C_H$.
  \item Run $S$ on $C_n$ to get $T_{C_n}$, the transcript which results from the simulation, including $(\psi_1, \psi_2)$, a pair of injective functions describing a mapping from $C_n$ to $C_H$.
  \item Output $T_G$ and $T_{C_n}$ along with $(\Phi_1, \Phi_2)$, defined to be $(\phi_1^{-1}\circ\psi_1, \phi_2^{-1}\circ\psi_2)$.
  \end{enumerate}
  Note that the output of this function is a pair of injective functions describing a mapping from $C_n$ to $G$.
  Since $\phi_1$ and $\psi_1$ are both injective and both map $n$ vertices onto the $n$ rows of $H$, $\phi_1^{-1}\circ\psi_1$ is an injective mapping from the vertices of $C_n$ to the vertices of $G$.
  The same reasoning applies to $\phi_2^{-1}\circ\psi_2$ and the columns of $H$.
  These mappings compose (with high probability) because $S$ simulates (with high probability) the prover using the same common reference string twice.

  Construct the decision algorithm $A$ as follows on input $G$:
  \begin{enumerate}
  \item Run algorithm $A'$ on input $G$ to yield $(\Phi_1, \Phi_2)$ (ignore the rest of the transcripts).
  \item 
    Check that $(\Phi_1, \Phi_2)$ describes a Hamiltonian cycle in $G$.
    Specifically, accept if and only if
    \begin{displaymath}
      \left\{(\Phi_1(i), \Phi_2(i \pmod{n} + 1)) \,\middle|\, i \in \{1, 2, \ldots, n\}\right\}\subseteq E.
    \end{displaymath}
  \end{enumerate}

  We claim that $A$ runs in probabilistic polynomial time.
  The running time of the first two steps of $A$ is polynomial in the size of the graph $G$ because $S$ is a probabilistic polynomial time machine.
  Computing the images under $(\Phi_1, \Phi_2)$ of $1$ through $n$ can be performed in polynomial time.
  Checking that each pair $(\Phi_1(i), \Phi_2(i \bmod{n} + 1))$ is in $E$ can be done in polynomial time.
  Since each step of the algorithm can be performed in probabilistic polynomial time and there are a constant number of steps, the algorithm $A$ runs in probabilistic polynomial time.
  It remains to bound the error probabilities of $A$.

  First suppose that $G\in\lang{HAMCYCLE}$, so $G$ has a Hamiltonian cycle.
  $C_n\in\lang{HAMCYCLE}$ as well.
  Let $(\pi_1, \pi_2)$ and $(\sigma_1, \sigma_2)$ be the pairs of injective functions describing a mapping from $C_G$ to $C_H$ and $C_n$ to $C_H$, respectively, with high probability which are output by the $(P, V)$ protocol when run with common input $G$ and $C_n$, respectively.
  Suppose $(\phi_1, \phi_2)$ and $(\psi_1, \psi_2)$ are outputs of $S$ as described in the function $A'$.
  By the zero-knowledge property of the Feige--Shamir protocol, $(\pi_1, \pi_2)\cid (\phi_1, \phi_2)$ and $(\sigma_1, \sigma_2)\cid(\psi_1, \psi_2)$.
  Now $(\pi_1^{-1}\circ\sigma_1, \pi_2^{-1}\circ\sigma_2)$ and $(\phi_1^{-1}\circ\psi_1, \phi_2^{-1}\circ\psi_2)$ are pairs of injective functions, both mapping $C_n$ to $C_G$ with high probability (and the mappings compose because they each have domain and range of size $n$), and further $(\pi_1^{-1}\circ\sigma_1, \pi_2^{-1}\circ\sigma_2) \cid (\phi_1^{-1}\circ\psi_1, \phi_2^{-1}\circ\psi_2)$.

  We want to show that $\Pr[S\text{ accepts}]\geq 1-\eta(m)$ where $\eta$ is a negligible function and $m$ is the security parameter.
  For brevity, let $X$ be the event ``$(\phi_1^{-1}\circ\psi_1, \phi_2^{-1}\circ\psi_2)$ describes a cycle in $G$'' (so $\Pr[S\text{ accepts}]=\Pr[X]$) and $Y$ be the event ``$(\pi_1^{-1}\circ\psi_1, \pi_2^{-1}\circ\psi_2)$ describes a cycle in $G$''.
  Since $X\cid Y$ by the last sentence of the previous paragraph and since $Pr[Y]\geq 1-\nu(m)$ for some negligible function $\nu$ by the completeness of the $(P, V)$ protocol, we have
  \begin{align*}
    & \left|\Pr[X] - \Pr[Y]\right| \leq \nu(m) \\
    \implies & \Pr[X] \geq \Pr[Y] - \nu(m) \\
    \implies & \Pr[X] \geq 1 - \eta(m) - \nu(m) \\
    \implies & \Pr[X] \geq 1 - (\eta(m) + \nu(m))
  \end{align*}
  where $\nu$ is some negligible function which comes from the computational indistinguishability of the events $X$ and $Y$.
  Since the sum of two negligible functions is itself a negligible function, $\Pr[X]$ is bounded below by $1-\kappa(n)$ where $\kappa(n)$ is a negligible function.
  Event $X$ occurs exactly when algorithm $A$ accepts, so $A$ accepts with high probability.

  Suppose now that $G\notin\lang{HAMCYCLE}$, so no permutation of the vertices of $G$ is a Hamiltonian cycle.
  We wish to show that algorithm $A$ accepts with low probability.
  Assume that $A$ accepts $G$ with non-negligible probability with the intention of producing a contradiction.
  If $A$ accepts $G$ with non-negligible probability it follows from the construction of $A$ that $A'$ outputs a pair $(\Phi_1, \Phi_2)$ which defines a mapping from $C_n$ to a ``cycle'' in $G$ with non-negligible probability.
  Since $(\psi_1, \psi_2)$ describes a mapping from $C_n$ to $C_H$ correctly with high probability (because $C_n\in\lang{HAMPATH}$), the faulty part of this mapping is $(\phi_1^{-1}, \phi_2^{-1})$.
  Hence $(\phi_1, \phi_2)$ is the certificate which will cause the verifier to accept with non-negligible probability.

  We will construct a cheating prover which uses $A'$ to fool $V$ into accepting $G$ with non-negligible probability.
  The prover proceeds as follows on input $G$:
  \begin{enumerate}
  \item
    Run $A'$ to produce $T_G$, $T_{C_n}$, and $(\Phi_1, \Phi_2)$.
  \item Forward the appropriate parts of $T_G$ to the verifier.
  \end{enumerate}
  Running the $(P, V)$ protocol with this cheating prover on input $G$ will cause the prover to send the certificate $(\phi_1, \phi_2)$.
  As stated in the previous paragraph, the verifier will accept this string with non-negligible probability.

\item I don't know.
\item
  The proof is by contradiction using a hybrid argument.
  Assume $(P, V)$ is not a $k$-proof adaptive non-interactive witness indistinguishable protocol.
  Therefore there exists a probabilistic polynomial time verifier $V^*$, an auxiliary input $z$, a negligible function $\eta$, and a sufficiently large $n$ such that $\Pr[b=b'] > \frac{1}{2} + \eta(n)$.
  Define experiment $exp\mbox{-}k$ the instance of the game in which $b=1$ and $exp\mbox{-}0$ the instance of the game in which $b=0$.
  Suppose each iteration of the loop in the game is numbered from $1$ up to $k$.
  Construct intermediate hybrid experiments $exp\mbox{-}i$ in which the prover receives $w_0$ in each round up to and including round $i$ and $w_1$ in each round after round $i$, where $i\in\{0, 1, \ldots, k\}$.
  (Note that when $i=0$, $exp\mbox{-}i$ corresponds to an instance of the game in which $b$ was chosen to be $0$, and when $i=k$, $exp\mbox{-}i$ corresponds to an instance of the game in which $b$ was chosen to be $1$.)

  By a standard averaging argument, there exists an $i\in\{0,1,\ldots, k-1\}$ such that
  \begin{displaymath}
    \left{\big|}\Pr\left[V^*\text{ accepts in } exp\mbox{-}i\right] - \Pr\left[V^*\text{ accepts in } exp\mbox{-}(i+1)\right]\right{\big|} > \frac{\eta(n)}{k}.
  \end{displaymath}
  The only difference between $exp\mbox{-}i$ and $exp\mbox{-}(i+1)$ is that $V^*$ receives the output of $P(x, r, w_0)$ in the former and the output of $P(x, r, w_1)$ in the latter.
  Now if $p_0=P(x, r, w_0)$ and $p_1=P(x, r, w_1)$, both occurring in round $i$, then
  \begin{displaymath}
    \left{\big|}\Pr\left[V^*(x, z, r, p_0)\text{ accepts}\right] - \Pr\left[V^*(x, z, r, p_1)\text{ accepts}\right]\right{\big|} > \frac{\eta(n)}{k}.
  \end{displaymath}
  Since $\frac{\eta(k)}{n}$ is a negligible function, $V^*$ can distinguish whether $P$ knows $w_0$ or $w_1$.
  This is a contradiction with the hypothesis that $(P, V)$ is a $1$-proof adaptive non-interactive witness indistinguishable protocol.
  Therefore we conclude that $(P, V)$ is a $k$-proof adaptive non-interactive witness indistinguishable protocol.
\end{enumerate}
\end{document}
